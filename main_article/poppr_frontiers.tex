%%%%%%%%%%%%%%%%%%%%%%%%%%%%%%%%%%%%%%%%%%%%%%%%%%%%%%%%%%%%%%%%%%%%%%%%%%%%%%%%%%%%%%%%%%%%%%%%%%%%%%%%%%%%%%%%%%%%%%%%%%%%%%%%%%%%%%%%%%%%%%%%%%%%%%%%%%%
% This is just an example/guide for you to refer to when submitting manuscripts to Frontiers, it is not mandatory to use frontiers.cls nor frontiers.tex  %
% This will only generate the Manuscript, the final article will be typeset by Frontier after acceptance.                                                 %
%                                                                                                                                                         %
% When submitting your files, remember to upload this *tex file, the pdf generated with it, the *bib file (if bibliography is not within the *tex) and all the figures.
%%%%%%%%%%%%%%%%%%%%%%%%%%%%%%%%%%%%%%%%%%%%%%%%%%%%%%%%%%%%%%%%%%%%%%%%%%%%%%%%%%%%%%%%%%%%%%%%%%%%%%%%%%%%%%%%%%%%%%%%%%%%%%%%%%%%%%%%%%%%%%%%%%%%%%%%%%%

%%% Version 3.0 Generated 2014/12/19 %%%
%%% You will need to have the following packages installed: datetime, fmtcount, etoolbox, fcprefix, which are normally inlcuded in WinEdt. %%%
%%% In http://www.ctan.org/ you can find the packages and how to install them, if necessary. %%%

\documentclass{frontiersSCNS} % for Science, Engineering and Humanities and Social Sciences articles
%\documentclass{frontiersHLTH} % for Health articles
%\documentclass{frontiersFPHY} % for Physics articles

%\setcitestyle{square}
\usepackage{url,lineno}
\linenumbers


% BELOW TAKEN FROM rticles plos template
%
% amsmath package, useful for mathematical formulas
\usepackage{amsmath}
% amssymb package, useful for mathematical symbols
\usepackage{amssymb}

% hyperref package, useful for hyperlinks
\usepackage{hyperref}

% graphicx package, useful for including eps and pdf graphics
% include graphics with the command \includegraphics
\usepackage{graphicx}

% Sweave(-like)
\usepackage{fancyvrb}
\DefineVerbatimEnvironment{Sinput}{Verbatim}{fontshape=sl}
\DefineVerbatimEnvironment{Soutput}{Verbatim}{}
\DefineVerbatimEnvironment{Scode}{Verbatim}{fontshape=sl}
\newenvironment{Schunk}{}{}
\DefineVerbatimEnvironment{Code}{Verbatim}{}
\DefineVerbatimEnvironment{CodeInput}{Verbatim}{fontshape=sl}
\DefineVerbatimEnvironment{CodeOutput}{Verbatim}{}
\newenvironment{CodeChunk}{}{}

% cite package, to clean up citations in the main text. Do not remove.
\usepackage{cite}

\usepackage{color}

% Below is from frontiers
%
\bibliographystyle{frontiersinSCNS}
% Use doublespacing - comment out for single spacing
%\usepackage{setspace}
%\doublespacing


% Leave a blank line between paragraphs instead of using \\



\def\keyFont{\fontsize{8}{11}\helveticabold }

%% ** EDIT HERE **
%% PLEASE INCLUDE ALL MACROS BELOW

%% END MACROS SECTION



  \def\Authors{
  Zhian N. Kamvar\,\textsuperscript{1},
  Jonah C. Brooks\,\textsuperscript{2},
  Niklaus J. Grunwald\,\textsuperscript{1,3*}}
% \\
\def\Address{
  \textsuperscript{1} Botany and Plant Pathology, Oregon State University,  Corvallis,  OR,  USA\\
  \textsuperscript{2} College of Electrical Engineering and Computer Science, Oregon State University,  Corvallis,  OR,  USA\\
  \textsuperscript{3} Horticultural Crops Research Laboratory, USDA-Agricultural Research Service,  Corvallis,  OR,  USA\\
}

  
  \def\firstAuthorLast{Kamvar {et~al.}}
  
  
  \def\corrAuthor{Niklaus J. Grunwald}\def\corrAddress{Horticultural Crops Research Laboratory USDA ARS\\3420 NW Orchard Ave.\\Corvallis, OR, 97330}\def\corrEmail{\href{mailto:grunwaln@science.oregonstate.edu}{\nolinkurl{grunwaln@science.oregonstate.edu}}}
  


\begin{document}
\onecolumn
\firstpage{1}

\title[Tool time]{Tools for detection and analysis of clonal genotypes in genome-wide SNP
data}
\author[\firstAuthorLast]{\Authors}
\address{}
\correspondance{}
\extraAuth{}% If there are more than 1 corresponding author, comment this line and uncomment the next one.
%\extraAuth{corresponding Author2 \\ Laboratory X2, Institute X2, Department X2, Organization X2, Street X2, City X2 , State XX2 (only USA, Canada and Australia), Zip Code2, X2 Country X2, email2@uni2.edu}
\topic{}% If your article is part of a Research Topic, please indicate here which.
\maketitle

\begin{abstract}

Understanding the dynamics of plant pathogen populations is imperative to
understanding plant microbe interactions, adaptation of pathogens to hosts,
reactive breeding and deployment of R genes and design of informed management
strategies. With the advent of high throughput sequencing technologies,
obtaining genomic sequences for representative populations has been easier than
ever before. This in combination with a move towards more reproducible research
has led researchers to move away from traditional stand alone programs for
population genetic analysis to modular programs such as R, which has a vibrant
community of contributers submitting packages tailored for several analyses,
including population genetics. The R package poppr was released to specifically
address issues surrounding analysis of clonal populations and has recently
implemented several tools that better facilitate the analysis of clonal
populations across different levels of hierarchies, implementing multilocus
genotype definitions and the index of association for reduced representation
genomic data, and modular bootstrapping of any genetic distance.

\tiny
 \keyFont{ \section{Keywords:} clonality, population genetics, bootstrap, open source}
\end{abstract}

\section*{Introduction}\label{introduction}
\addcontentsline{toc}{section}{Introduction}

Most computational tools for population genetics are based on concepts
developed for sexual model organisms. Populations that reproduce
clonally or are polyploid are thus difficult to characterize using
classical population genetic tools.

The research community using the R statistical and computing language
(RCoreTeam, 2013) has developed a plethora of new resources for
population genetic analysis (Paradis, 2010; Jombart, 2008). Recently, we
introduced the R package \emph{poppr} specifically developed for
analysis of clonal populations (Kamvar et al., 2014). \emph{Poppr}
previously introduced several novel features including the ability to
conduct a hierarchical analysis across unlimited hierarchies, test for
linkage association, graph minimum spanning networks or provide
bootstrap support for Bruvo's distance in resulting trees. In its first
iteration, \emph{poppr} was appropriate for traditional markers systems,
but not ideally suited to population genomic data resulting from high
throughput sequencing methods as increases in the number of markers
decreased the ability to distinguish clonal genotypes due to higher
chance to detect somatic mutations or genotyping error.

\section*{Materials and Methods}\label{materials-and-methods}
\addcontentsline{toc}{section}{Materials and Methods}

In \emph{poppr} version 1.1, a new S4 object was defined to expand the
genind object of \emph{adegenet}. The genclone object formalized the
definitions of multilocus genotypes and population hierarchies by adding
two slots called mlg and hierarchy that carried a numeric vector and a
data frame, respectively. These new slots allow for increased efficiency
and ease of use by allowing these metadata to travel with the genetic
data.

\subsection*{Clone correction}\label{clone-correction}
\addcontentsline{toc}{subsection}{Clone correction}

As highlighted in previous work, clone correction is an important part
of population genetic analysis of organisms that have cryptic growth or
are known to produce asexual propagules (Kamvar et al., 2014; Milgroom,
1996; Grünwald et al., 2003). This method removes bias that would
otherwise affect metrics that rely on allele frequencies. It was
initially designed for data with only a handful of markers. With the
advent of large-scale sequencing and reduced- representation libraries,
it has become easier to sequence tens of thousands of markers from
hundreds of individuals (Elshire et al., 2011; Davey et al., 2011; Davey
and Blaxter, 2010). With this number of markers, the genetic resolution
is much greater, but the chance of genotyping error is also greatly
increased (Mastretta-Yanes et al., 2014). Taking this and somatic
mutation into account, it would be impossible to separate true clones
from independent individuals by just comparing what multilocus genotypes
are different. We introduce a a new transparent method for collapsing
unique multilocus genotypes determined by naive string comparison into
multilocus lineages utilizing any genetic distance given three different
clustering algorithms: farthest neighbor, nearest neighbor, and UPGMA
(average neighbor) {[}citation needed{]}.

The clustering algorithms act on a distance matrix that is either
provided by the user or generated via a function that will calculate a
distance from genclone objects such as \texttt{bruvo.dist}, which will
calculate Bruvo's distance at any level of ploidy. All algorithms have
been implemented in C and utilize the open MP framework for optional
parallel processing. Default is the conservative farthest neighbor
algorithm, which will only cluster samples together only if all samples
in the cluster are at a distance less than the given threshold. By
contrast, the nearest neighbor algorithm will have a chaining effect
that will cluster samples akin to adding links on a chain where a sample
can be included in a cluster if all of the samples have at least one
connection below a given threshold. The UPGMA, or average neighbor
clustering algorithm is the one most familiar to biologists as it is
often used to generate preliminary ultra-metric trees based off of
genetic distance. This algorithm will cluster by creating a
representative sample per cluster and joining clusters if these
representative samples are closer than the given threshold.

\subsubsection*{\texorpdfstring{\emph{P.
infestans}}{P. infestans}}\label{p.-infestans}
\addcontentsline{toc}{subsubsection}{\emph{P. infestans}}

\emph{Phytophthora infestans} is the causal agent of potato late blight
originating from Mexico and spread to Europe in the mid 19th century
{[}Goss et al. (2014); cooke{]}. \emph{P. infestans} reproduces both
clonally and sexually. The clonal lineages of \emph{P. infestans} have
been formally defined into 18 separate clonal lineages using various
molecular methods. We utilize these data to show how the
\texttt{mlg.filter} function collapses multilocus genotypes with Bruvo's
distance assuming a genome addition model (Bruvo et al., 2004).

\subsection*{Index of association}\label{index-of-association}
\addcontentsline{toc}{subsection}{Index of association}

The index of association (\(I_A\)) is a measure of multilocus linkage
disequilibrium that is most often used to detect clonal reproduction
within organisms that have the ability to reproduce via sexual or
asexual processes (brown1980; maynard1993; Milgroom, 1996). It was
standardized in 2001 as \(\bar{r}_d\) by (Agapow and Burt, 2001) to
address the issue of scaling with increasing number of loci. This metric
is typically applied to traditional dominant and co-dominant markers
such as AFLPs, SNPs, or microsatellite markers. With the advent of high
throughput sequencing, SNP data is now available in in a genome-wide
context and in very large matrices including thousands of SNPs. Thus,
the likelihood of finding mutations within two individuals of a given
clone increases and tools are needed for defining clone boundaries. For
this reason, we devised two approaches using the index of association
for large numbers of markers typical for population genomic studies.

The first approach is a sliding window approach that would utilize the
position of markers in the genome to calculate \(\bar{r}_d\) among any
number of SNPs found within the windowed region. It is important that
this calculation utilize \(\bar{r}_d\) as the number of loci will be
different within each window (Agapow and Burt, 2001). This approach
would be suited for a quick calculation of linkage disequilibrium across
the genome that can detect potential hotspots of LD that could be
investigated further with more computationally intensive methods
assuming that the number of samples \textless{}\textless{} the number of
loci.

A sliding window approach would not be good for utilizing \(\bar{r}_d\)
as a test for clonal reproduction as it would necessarily focus on loci
within a short section of the genome that may or may not be recombining.
A remedy for this is to randomly sample \(m\) loci, calculate
\(\bar{r}_d\) and repeat \(r\) times, creating a distribution of
expected values of \(\bar{r}_d\).

As the data for this would be stored in a genlight object, which stores
genetic information in bits, the calculations of \(\bar{r}_d\) are
performed in parallelized bit level C code for efficiency (Jombart,
2008).

\subsection*{Population Strata and
Hierarchies}\label{population-strata-and-hierarchies}
\addcontentsline{toc}{subsection}{Population Strata and Hierarchies}

Assessments of population structure through methods such as hierarchical
\(F_{st}\) and AMOVA benefit greatly from multiple levels of population
definition (Linde et al., 2002; Everhart and Scherm, 2014). With clonal
organisms, basic practice has been to clone-censor data to avoid
downward bias in diversity due to duplicated genotypes that may or may
not represent different samples (Milgroom, 1996). Data structures for
population genetic data mostly allow for only one level of hierarchical
definition. The impetus was placed on the researchers to provide the
population hierarchies for every step of the analysis. In poppr version
1.1, the hierarchy slot was introduced to allow unlimited population
hierarchies or stratifications to travel with the data. In practice, it
is stored as a data frame where each column represents a separate
hierarchical level. The user could then define the population factor for
the data by providing a hierarchical formula that would contain one or
more levels from the hierarchy slot to be combined.

\subsection*{Genotype Accumulation
Curve}\label{genotype-accumulation-curve}
\addcontentsline{toc}{subsection}{Genotype Accumulation Curve}

Analysis of population genetics of clonal organisms often borrows from
ecological methods such as analysis of diversity within populations
(Milgroom, 1996; Arnaud-Hanod et al., 2007). When choosing markers for
analysis, it is important to make sure that the observed diversity in
your sample will not appreciably increase if an additional marker is
added (Arnaud-Hanod et al., 2007). This concept is analogous to a
species accumulation curve, obtained by rarefaction. The genotype
accumulation curve in \emph{poppr} is implemented by randomly sampling
\(x\) loci and counting the number of observed MLGs. This repeated \(r\)
times for 1 locus up to \(n-1\) loci, creating \(n-1\) distributions of
observed MLGs.

\subsection*{Minimum Spanning Networks}\label{minimum-spanning-networks}
\addcontentsline{toc}{subsection}{Minimum Spanning Networks}

Poppr introduced minimum spanning networks in it's original iteration
that were based off of igraph's function minimum.spanning.tree. This
algorithm would produce a minimum spanning tree, but with no
reticulations, which is common and expected of clonal organisms. In
other minimum spanning network programs, reticulation is obtained by
calculating the minimum spanning tree several times and returning the
set of all edges included in the trees. Due to the way igraph has
implemented Prim's algorithm, it is not possible to utilize this
strategy, thus we have implemented a C function to walk through a
distance matrix and minimum spanning tree to connect groups of nodes
with edges of equal weight.

\subsection*{Bootstrapping}\label{bootstrapping}
\addcontentsline{toc}{subsection}{Bootstrapping}

Calculating genetic distance for among samples and populations is very
important method for assessing population differentiation through
methods such as \(F_{st}\), AMOVA, and Mantel tests. Confidence in
distance metrics is related to the confidence in the markers to
accurately represent the diversity of the data. Especially true with
microsatellite markers, a single hyper-diverse locus can make a
population appear to have more diversity based on genetic distance.
Using a bootstrapping procedure of randomly sampling loci with
replacement when calculating a distance matrix gives confidence in
hierarchical clustering. Because genetic data in a genind object is
represented as a matrix with samples in rows and alleles in columns,
bootstrapping is a non-trivial task as all alleles in a single locus
need to be sampled together. To remedy this, we have created an internal
S4 class called ``bootgen'', which extends the internal ``gen'' class
from adegenet. This class can be created from any genind, genclone, or
genpop object, and allows loci to be sampled with replacement. To
further facilitate bootstrapping, a function called aboot, which stands
for ``any boot'', is introduced that will bootstrap any genclone,
genind, or genpop object with any genetic distance that can be
calculated from it.

\subsection*{Additions}\label{additions}
\addcontentsline{toc}{subsection}{Additions}

\begin{itemize}
\itemsep1pt\parskip0pt\parsep0pt
\item
  Psex
\item
  Bootstrapping for individuals or populations for any genetic distance
\item
  MLG clustering
\item
  Index of Association for genomic sequences
\item
  Parallelization
\item
  Genotype Accumulation Curve
\item
  Matching Ties in Minimum Spanning networks (Cite ramorum paper)
\end{itemize}

\section*{Results}\label{results}
\addcontentsline{toc}{section}{Results}

\subsection*{Clonal identification}\label{clonal-identification}
\addcontentsline{toc}{subsection}{Clonal identification}

\subsubsection*{\texorpdfstring{\emph{P.
infestans}}{P. infestans}}\label{p.-infestans-1}
\addcontentsline{toc}{subsubsection}{\emph{P. infestans}}

The three algorithms were able to detect 18 multilocus lineages at
different distances (Fig. 1). Contingency tables between the described
multilocus genotypes and the genotypes defined by distance show that
most of the 18 lineages were resolved, except for US-8, which is
polytomic (Table 1).

\begin{quote}
Nik, these are just to see how the figures look in the paper, the actual
figures need to be uploaded separately, according to the frontiers
specifications.
\end{quote}

\begin{CodeChunk}

\includegraphics{poppr_frontiers_files/figure-latex/pinf_data-1} \end{CodeChunk}\begin{CodeChunk}

\includegraphics{poppr_frontiers_files/figure-latex/pinf_tree-1} \end{CodeChunk}\begin{table}[ht]
\centering
\begin{tabular}{ccccccccccccccccccc}
  \hline
 & 1 & 3 & 5 & 8 & 9 & 10 & 12 & 13 & 15 & 17 & 19 & 20 & 23 & 24 & 25 & 27 & 28 & 29 \\ 
  \hline
B & . & . & . & . & . & . & . & . & 1 & . & . & . & . & . & . & . & . & . \\ 
  C & . & 1 & . & . & . & . & . & . & . & . & . & . & . & . & . & . & . & . \\ 
  EU-13 & . & . & . & . & 1 & . & . & . & . & . & . & . & . & . & . & . & . & . \\ 
  EU-4 & . & . & . & . & . & 1 & . & . & . & . & . & . & . & . & . & . & . & . \\ 
  EU-8 & . & . & . & . & . & . & . & 1 & . & . & . & . & . & . & . & . & . & . \\ 
  EU-5 & . & . & . & . & . & . & 2 & . & . & . & . & . & . & . & . & . & . & . \\ 
  US-8 & . & . & . & . & . & . & . & . & . & . & . & 1 & . & 2 & 1 & . & . & . \\ 
  D.1 & 1 & . & . & . & . & . & . & . & . & . & . & . & . & . & . & . & . & . \\ 
  D.2 & 1 & . & . & . & . & . & . & . & . & . & . & . & . & . & . & . & . & . \\ 
  US-11 & . & . & . & . & . & . & . & . & . & 2 & . & . & . & . & . & . & . & . \\ 
  US-12 & . & . & 1 & . & . & . & . & . & . & . & . & . & . & . & . & . & . & . \\ 
  US-14 & . & . & . & . & . & . & . & . & . & . & . & . & . & 1 & . & . & . & . \\ 
  US-17 & . & . & . & . & . & . & . & . & . & . & . & . & . & . & . & . & 1 & . \\ 
  US-20 & . & . & . & . & . & . & . & . & . & . & . & . & . & . & . & 2 & . & . \\ 
  US-21 & . & . & . & . & . & . & . & . & . & . & 2 & . & . & . & . & . & . & . \\ 
  US-22 & . & . & . & . & . & . & . & . & . & . & . & . & . & . & . & . & . & 2 \\ 
  US-23 & . & . & . & 3 & . & . & . & . & . & . & . & . & . & . & . & . & . & . \\ 
  US-24 & . & . & . & . & . & . & . & . & . & . & . & . & 3 & . & . & . & . & . \\ 
   \hline
\end{tabular}
\end{table}

\subsection*{Simulations}\label{simulations}
\addcontentsline{toc}{subsection}{Simulations}

100 simulations were run. We found that across all methods, detection of
duplicated samples had \(\sim\) 97\% true positive fraction and \(\sim\)
2\% false positive fraction indicating that this method is robust to
simulated populations.

\subsection*{Speed}\label{speed}
\addcontentsline{toc}{subsection}{Speed}

\subsection*{Community Acceptance}\label{community-acceptance}
\addcontentsline{toc}{subsection}{Community Acceptance}

\begin{itemize}
\itemsep1pt\parskip0pt\parsep0pt
\item
  articles that have cited poppr
\item
  hierarchical methods have been ported to adegenet
\end{itemize}

\section*{Discussion}\label{discussion}
\addcontentsline{toc}{section}{Discussion}

Reticulate minimum spanning networks are very important for extremely
clonal organisms where a minimum spanning tree would become a chain. The
current implementation in poppr has been successfully used in analyses
such as reconstruction of the \emph{P. ramorum} outbreak in Curry
County, OR {[}kamvar2015spatial{]}. Reticulated networks also allow for
the addition of graph community detection algorithms such as the infomap
algorithm, giving the user another tool to detect clonal or population
structure (Rosvall and Bergstrom, 2008). While it is possible to utilize
these graph walking algorithms on non-reticulate minimum spanning trees,
information that would be used by the walkers is not present in the
graph and nodes that would truly be connected might be defined as
separate communities.

\begin{itemize}
\itemsep1pt\parskip0pt\parsep0pt
\item
  loops in graphs allow for analysis of population structure via graph
  walking algorithms in igraph
\item
  bootstrapping methods encourage future developers to write distance
  implementations in common format
\item
  moving towards open source, modular tools is the direction that
  population genetics and plant pathology needs to go.
\end{itemize}

\section*{References}\label{references}
\addcontentsline{toc}{section}{References}

Agapow, P.-M., and Burt, A. (2001). Indices of multilocus linkage
disequilibrium. \emph{Molecular Ecology Notes} 1, 101--102.
doi:\href{http://dx.doi.org/10.1046/j.1471-8278.2000.00014.x}{10.1046/j.1471-8278.2000.00014.x}.

Arnaud-Hanod, S., Duarte, C. M., Alberto, F., and Serr{ã}o, E. A.
(2007). Standardizing methods to address clonality in population
studies. \emph{Molecular Ecology} 16, 5115--5139.

Bruvo, R., Michiels, N. K., D'Souza, T. G., and Schulenburg, H. (2004).
A simple method for the calculation of microsatellite genotype distances
irrespective of ploidy level. \emph{Molecular Ecology} 13, 2101--2106.

Davey, J. W., and Blaxter, M. L. (2010). RADSeq: Next-generation
population genetics. \emph{Briefings in Functional Genomics} 9,
416--423.
doi:\href{http://dx.doi.org/10.1093/bfgp/elq031}{10.1093/bfgp/elq031}.

Davey, J. W., Hohenlohe, P. A., Etter, P. D., Boone, J. Q., Catchen, J.
M., and Blaxter, M. L. (2011). Genome-wide genetic marker discovery and
genotyping using next-generation sequencing. \emph{Nature Reviews
Genetics} 12, 499--510.

Elshire, R. J., Glaubitz, J. C., Sun, Q., Poland, J. A., Kawamoto, K.,
Buckler, E. S., and Mitchell, S. E. (2011). A robust, simple
genotyping-by-sequencing (gBS) approach for high diversity species.
\emph{PloS one} 6, e19379.

Everhart, S., and Scherm, H. (2014). Fine-scale genetic structure of
monilinia fructicola during brown rot epidemics within individual peach
tree canopies. \emph{Phytopathology}.

Goss, E. M., Tabima, J. F., Cooke, D. E., Restrepo, S., Fry, W. E.,
Forbes, G. A., Fieland, V. J., Cardenas, M., and Gr{ü}nwald, N. J.
(2014). The irish potato famine pathogen phytophthora infestans
originated in central mexico rather than the andes. \emph{Proceedings of
the National Academy of Sciences} 111, 8791--8796.

Grünwald, N. J., Goodwin, S. B., Milgroom, M. G., and Fry, W. E. (2003).
Analysis of genotypic diversity data for populations of microorganisms.
\emph{Phytopathology} 93, 738--46. Available at:
\url{http://apsjournals.apsnet.org/doi/abs/10.1094/PHYTO.2003.93.6.738}.

Jombart, T. (2008). Adegenet: a R package for the multivariate analysis
of genetic markers. \emph{Bioinformatics} 24, 1403--1405.
doi:\href{http://dx.doi.org/10.1093/bioinformatics/btn129}{10.1093/bioinformatics/btn129}.

Kamvar, Z. N., Tabima, J. F., and Gr{ü}nwald, N. J. (2014). Poppr: An r
package for genetic analysis of populations with clonal, partially
clonal, and/or sexual reproduction. \emph{PeerJ} 2, e281.

Linde, C., Zhan, J., and McDonald, B. (2002). Population structure of
mycosphaerella graminicola: From lesions to continents.
\emph{Phytopathology} 92, 946--955.

Mastretta-Yanes, A., Zamudio, S., Jorgensen, T. H., Arrigo, N., Alvarez,
N., Pi{ñ}ero, D., and Emerson, B. C. (2014). Gene duplication,
population genomics, and species-level differentiation within a tropical
mountain shrub. \emph{Genome biology and evolution} 6, 2611--2624.

Milgroom, M. G. (1996). Recombination and the multilocus structure of
fungal populations. \emph{Annual Review of Phytopathology} 34, 457--477.

Paradis, E. (2010). Pegas: an R package for population genetics with an
integrated--modular approach. \emph{Bioinformatics} 26, 419--420.

RCoreTeam (2013). R: A Language and Environment for Statistical
Computing. Available at: \url{http://www.R-project.org/}.

Rosvall, M., and Bergstrom, C. T. (2008). Maps of random walks on
complex networks reveal community structure. \emph{Proceedings of the
National Academy of Sciences} 105, 1118--1123.


\end{document}

